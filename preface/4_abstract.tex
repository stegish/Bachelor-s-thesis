\cleardoublepage
\phantomsection
\pdfbookmark{Compendio}{Compendio}
\begingroup
\let\clearpage\relax
\let\cleardoublepage\relax
\chapter*{Sommario}

La presente tesi illustra l’ideazione, la progettazione e la prototipazione di una piattaforma software 
integrata in un gestionale orientata alle \gls{pmi}, concepita per ottimizzare i processi produttivi 
attraverso un’integrazione fra servizi \textit{cloud\nobreakdash-native}, interfacce mobile e funzionalità 
di raccomandazione basate su modelli di \textit{IA}.  
L’elaborato percorre l’intero ciclo di vita del progetto: dall’analisi dei requisiti funzionali e non 
funzionali alla definizione dell’architettura, fino alla validazione sperimentale dei risultati ottenuti.

La scelta di concentrarsi su una soluzione fortemente applicativa nasce da un’esperienza maturata in 
contesti eterogenei: dapprima lavorando all’interno di una \gls{pmi} per 3 anni, quindi in una \textit{startup} impegnata 
nello sviluppo di un gestionale innovativo. In entrambe le realtà è emersa con chiarezza la necessità 
di strumenti agili, facilmente scalabili e in grado di generare valore tangibile sui flussi di lavoro 
quotidiani.  
Collaboro con \textit{Devess~S.r.l.} da oltre un anno e, proprio durante questa esperienza, 
ho concepito la funzionalità di monitoraggio intelligente descritta nella tesi.

Partendo da queste considerazioni, la tesi si propone di dimostrare la concreta fattibilità tecnica 
del prototipo realizzato con un \textit{back\nobreakdash-end} \textit{Python} e un 
\textit{front\nobreakdash-end} Flutter e la sua utilità potenziale nell’incrementare l’efficienza 
operativa. Il lavoro discute inoltre le prospettive di evoluzione futura della piattaforma e i benefici 
che un’adozione su larga scala potrebbe apportare all’ecosistema produttivo di un'azienda.

\endgroup
\vfill
