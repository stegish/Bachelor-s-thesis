% -------------------- Acronimi --------------------
\newacronym{api}{API}{Application Programming Interface}
\newacronym{sdk}{SDK}{Software Development Kit}
\newacronym{uml}{UML}{Unified Modeling Language}
\newacronym{pmi}{PMI}{Piccola e Media Impresa}
\newacronym{poc}{PoC}{Proof of Concept}
\newacronym{kpi}{KPI}{Key Performance Indicator}
\newacronym{llm}{LLM}{Large Language Model}
\newacronym{mcp}{MCP}{Model Context Protocol}
\newacronym{csv}{CSV}{Comma-Separated Values}


% -------------------- Voci del Glossario --------------------
\newglossaryentry{agile}{
    name={agile},
    sort=agile,
    description={Approccio allo sviluppo software e alla gestione dei progetti basato su cicli di lavoro iterativi e incrementali. Le metodologie agili, come Scrum, privilegiano la flessibilità, la collaborazione con il cliente e la capacità di rispondere rapidamente al cambiamento.}
}

\newglossaryentry{api_entry}{
    name={API (Application Programming Interface)},
    sort=api,
    description={Insieme di regole, specifiche e strumenti che definiscono come i componenti software debbano interagire tra loro. Un’API espone un’interfaccia attraverso la quale un’applicazione può richiedere servizi, dati o funzionalità a un’altra, nascondendo i dettagli di implementazione interni.}
}

\newglossaryentry{backend}{
    name={back-end},
    sort=backend,
    description={La parte di un'applicazione software che non è direttamente accessibile dall'utente finale. Comprende la logica di business, l'interazione con i database, le API e l'elaborazione dei dati. È il "motore" del sistema, in contrapposizione al \gls{frontend}.}
}

\newglossaryentry{csv_entry}{
    name={CSV (Comma-Separated Values)},
    sort=csv,
    description={Formato di file testuale utilizzato per l'archiviazione di dati in forma tabellare. Ogni riga del file rappresenta una riga della tabella e i campi al suo interno sono separati da un carattere delimitatore, tipicamente una virgola.}
}

\newglossaryentry{diffare}{
    name={diffare},
    sort=diffare,
    description={Verbo del gergo informatico, derivato dal comando Unix \texttt{diff}. Significa confrontare due versioni di un file di testo o di un codice sorgente per evidenziarne le differenze. È una pratica fondamentale nei sistemi di controllo di versione come Git.}
}

\newglossaryentry{flutter}{
    name={Flutter},
    sort=flutter,
    description={Toolkit UI open-source sviluppato da Google per la creazione di interfacce utente performanti e visivamente coerenti per applicazioni mobile, web e desktop, partendo da un'unica base di codice in linguaggio Dart.}
}

\newglossaryentry{framework}{
    name={framework},
    sort=framework,
    description={In informatica, una struttura software di base che fornisce funzionalità generiche e un'architettura predefinita. Gli sviluppatori estendono il framework con codice specifico per l'applicazione, accelerando lo sviluppo e promuovendo pratiche standard.}
}

\newglossaryentry{frontend}{
    name={front-end},
    sort=frontend,
    description={La parte di un'applicazione software con cui l'utente interagisce direttamente, nota anche come interfaccia utente (UI). Comprende gli elementi visivi (pulsanti, menu, grafici) e la logica di presentazione dei dati.}
}

\newglossaryentry{kpi_entry}{
    name={KPI (Key Performance Indicator)},
    sort=kpi,
    description={Metrica quantitativa utilizzata per valutare l’efficacia di un processo o di un’attività aziendale in rapporto agli obiettivi prefissati, supportando decisioni basate su dati.}
}

\newglossaryentry{langchain}{
    name={LangChain},
    sort=langchain,
    description={Framework open-source per lo sviluppo di applicazioni basate su modelli linguistici (LLM). Semplifica l'orchestrazione di catene complesse (\textit{chains}), la gestione della memoria conversazionale e l'integrazione con fonti dati e strumenti esterni.}
}

\newglossaryentry{leadtime}{
    name={lead time},
    sort=lead time,
    description={Il termine \textit{lead time} indica l'intervallo di tempo che intercorre tra l'avvio e il completamento di un processo produttivo o logistico, utilizzato per misurare l'efficienza operativa e i tempi di risposta del sistema.}
}

\newglossaryentry{llm_entry}{
    name={LLM (Large Language Model)},
    sort=llm,
    description={Modello di intelligenza artificiale basato su architetture di reti neurali profonde, addestrato su vasti corpora di testo per comprendere, generare e manipolare il linguaggio umano in modo coerente e contestualmente appropriato.}
}

\newglossaryentry{log}{
    name={log},
    sort=log,
    description={In ambito informatico, il termine \textit{log} indica un registro cronologico di eventi, messaggi o transazioni generati da un sistema, applicazione o dispositivo, utilizzato per il monitoraggio, la diagnostica e l’audit.}
}

\newglossaryentry{mcp_entry}{
    name={MCP (Model Context Protocol)},
    sort=mcp,
    description={Protocollo custom che definisce le regole di interazione tra un Large Language Model (LLM) e strumenti esterni (\textit{tools}). Specifica il vocabolario degli strumenti disponibili, la sintassi per la loro invocazione e gli schemi per la validazione dei dati.}
}

\newglossaryentry{middleware}{
    name={middleware},
    sort=middleware,
    description={Software che agisce come intermediario tra due o più componenti applicativi, facilitandone la comunicazione e l'interoperabilità. Nel contesto di questo lavoro, si riferisce al livello logico che orchestra la comunicazione tra l'agente LLM e gli strumenti esterni.}
}

\newglossaryentry{nosql}{
    name={NoSQL},
    sort=nosql,
    description={Famiglia di sistemi di gestione di basi di dati non relazionali, progettati per offrire elevata scalabilità e flessibilità. I modelli più comuni includono database a documenti (come MongoDB), a colonne, chiave-valore e a grafo.}
}

\newglossaryentry{pmi_entry}{
    name={PMI (Piccola e Media Impresa)},
    sort=pmi,
    description={Impresa che, ai sensi della Raccomandazione 2003/361/CE della Commissione europea, impiega meno di 250 addetti e realizza un fatturato annuo non superiore a 50\,milioni di euro oppure un totale di bilancio annuo non superiore a 43\,milioni di euro.}
}

\newglossaryentry{poc_entry}{
    name={PoC (Proof of Concept)},
    sort=poc,
    description={Realizzazione pratica e preliminare di un'idea o di un metodo per dimostrarne la fattibilità e verificarne il potenziale. Un PoC non è un prototipo, ma una prova che il concetto fondamentale funziona.}
}

\newglossaryentry{sdk_entry}{
    name={SDK (Software Development Kit)},
    sort=sdk,
    description={Insieme di strumenti di sviluppo software che semplifica la creazione di applicazioni per una specifica piattaforma o linguaggio di programmazione, fornendo librerie, documentazione, esempi di codice e altri tool.}
}

\newglossaryentry{uml_entry}{
    name={UML (Unified Modeling Language)},
    sort=uml,
    description={Linguaggio di modellazione standardizzato utilizzato nell'ingegneria del software per visualizzare, specificare, costruire e documentare gli artefatti di un sistema software.}
}