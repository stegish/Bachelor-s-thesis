% Acronyms
\newacronym{api}{API}{Application Programming Interface}
\newacronym{sdk}{SDK}{Software Development Kit}
\newacronym{uml}{UML}{Unified Modeling Language}
\newacronym{tsa}{TSA}{Termine solo acronimo}
\newacronym{pmi}{PMI}{Piccole e Medie Imprese}

% Glossary
\newglossaryentry{apig}{
    name={API},
    text={Application Program Interface},
    sort=api,
    description={In informatics, an API is a set of procedures available to programmers, typically grouped to form a toolkit for a specific task within a program. Its purpose is to provide an abstraction, usually between hardware and the programmer or between low-level and high-level software, simplifying the programming process}
}

\newglossaryentry{sdkg}{
    name={SDK},
    text={Software Development Kit},
    sort=sdk,
    description={A Software Development Kit (SDK) is a collection of development tools in one installable package, facilitating application creation by providing a compiler, debugger, and sometimes a software framework. SDKs are typically specific to a hardware platform and operating system combination. Many application developers use specific SDKs to enable advanced functionalities such as advertisements, push notifications, etc}
}

\newglossaryentry{umlg}{
    name={UML},
    text={Unified Modeling Language},
    sort=uml,
    description={In software engineering, Unified Modeling Language (UML) is a modeling and specification language based on the object-oriented paradigm. UML serves as a "lingua franca" in the object-oriented design and programming community. Much of the industry literature uses UML to describe analytical and design solutions in a concise and understandable way for a broad audience}
}

\newglossaryentry{TermineSenzaAcronimo}{
    name={Nome del termine},
    sort=termine senza acronimo,
    description={Descrizione}
}

\newglossaryentry{LLM}{
    name={LLM},
    text={Large Language Model},
    sort=llm,
    description={Modello di intelligenza artificiale basato su architetture di reti neurali profonde, progettato per elaborare, comprendere e generare testo in linguaggio naturale. La loro capacità predittiva e generativa si migliora proporzionalmente alle dimensioni del modello e alla quantità di dati utilizzati durante la fase di training}
}

\newglossaryentry{API}{
    name={API},
    text={Application Programming Interface},
    sort=api,
    description={Insieme di regole, specifiche e strumenti che definiscono come i componenti software debbano interagire tra loro. Un’API espone un’interfaccia attraverso la quale un’applicazione può richiedere servizi, dati o funzionalità a un’altra, nascondendo i dettagli di implementazione interni}
}

\newglossaryentry{leadtime}{
    name={lead time},
    sort=lead time,
    description={Il termine \textit{lead time} indica l'intervallo di tempo che intercorre tra l'avvio e il completamento di un processo produttivo o logistico, utilizzato per misurare l'efficienza operativa e i tempi di risposta del sistema.}
}

\newglossaryentry{PoC}{
    name={PoC},
    text={Proof of Concept},
    sort=poc,
    description={Versione preliminare di un progetto o prototipo realizzato al fine di dimostrare la fattibilità di un’idea o di una tecnologia prima dello sviluppo completo.}
}

\newglossaryentry{KPI}{
    name={KPI},
    text={Key Performance Indicator},
    sort=kpi,
    description={Metrica quantitativa utilizzata per valutare l’efficacia di un processo o di un’attività aziendale in rapporto agli obiettivi prefissati, supportando decisioni basate su dati.}
}

\newglossaryentry{pmig}{
    name={PMI},
    text={Piccola e Media Impresa},
    sort=pmi,
    description={Impresa che, ai sensi della Raccomandazione 2003/361/CE della Commissione europea, impiega meno di 250 addetti e realizza un fatturato annuo non superiore a 50\,milioni di euro oppure un totale di bilancio annuo non superiore a 43\,milioni di euro.}
}

\newglossaryentry{log}{
    name={log},
    text={log},
    sort=log,
    description={In ambito informatico, il termine \textit{log} indica un registro cronologico di eventi, messaggi o transazioni generati da un sistema, applicazione o dispositivo, utilizzato per il monitoraggio, la diagnostica e l’audit.}
}
