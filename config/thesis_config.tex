% Load variables
\newcommand{\myUni}{Università degli Studi di Padova}
\newcommand{\myDepartment}{Dipartimento di Matematica ``Tullio Levi-Civita''}
\newcommand{\myFaculty}{Corso di Laurea in Informatica}
\newcommand{\myTitle}{Integrazione di Large Language Models e Data Analytics per l’Ottimizzazione delle Linee di Produzione Industriali.}
\newcommand{\myDegree}{Tesi di Laurea Triennale}
\newcommand{\profTitle}{Prof.}
\newcommand{\myProf}{Zanella Marco}
\newcommand{\graduateTitle}{Laureando}
\newcommand{\myName}{Fantinato Michael}
\newcommand{\myStudentID}{2043672}
\newcommand{\myAA}{2024-2025}
\newcommand{\myLocation}{Padova}
\newcommand{\myTime}{Luglio 2025}
% Acronyms
\newacronym{api}{API}{Application Program Interface}
\newacronym{sdk}{SDK}{Software Development Kit}
\newacronym{uml}{UML}{Unified Modeling Language}
\newacronym{tsa}{TSA}{Termine solo acronimo}
\newacronym{pmi}{PMI}{Piccole e Medie Imprese}

% Glossary
\newglossaryentry{apig}{
    name={API},
    text={Application Program Interface},
    sort=api,
    description={In informatics, an API is a set of procedures available to programmers, typically grouped to form a toolkit for a specific task within a program. Its purpose is to provide an abstraction, usually between hardware and the programmer or between low-level and high-level software, simplifying the programming process}
}

\newglossaryentry{sdkg}{
    name={SDK},
    text={Software Development Kit},
    sort=sdk,
    description={A Software Development Kit (SDK) is a collection of development tools in one installable package, facilitating application creation by providing a compiler, debugger, and sometimes a software framework. SDKs are typically specific to a hardware platform and operating system combination. Many application developers use specific SDKs to enable advanced functionalities such as advertisements, push notifications, etc}
}

\newglossaryentry{umlg}{
    name={UML},
    text={Unified Modeling Language},
    sort=uml,
    description={In software engineering, Unified Modeling Language (UML) is a modeling and specification language based on the object-oriented paradigm. UML serves as a "lingua franca" in the object-oriented design and programming community. Much of the industry literature uses UML to describe analytical and design solutions in a concise and understandable way for a broad audience}
}

\newglossaryentry{TermineSenzaAcronimo}{
    name={Nome del termine},
    sort=termine senza acronimo,
    description={Descrizione}
}

\newglossaryentry{LLM}{
    name={LLM},
    text={Large Language Model},
    sort=llm,
    description={Modello di intelligenza artificiale basato su architetture di reti neurali profonde, progettato per elaborare, comprendere e generare testo in linguaggio naturale. La loro capacità predittiva e generativa si migliora proporzionalmente alle dimensioni del modello e alla quantità di dati utilizzati durante la fase di training}
}

\newglossaryentry{API}{
    name={API},
    text={Application Programming Interface},
    sort=api,
    description={Insieme di regole, specifiche e strumenti che definiscono come i componenti software debbano interagire tra loro. Un’API espone un’interfaccia attraverso la quale un’applicazione può richiedere servizi, dati o funzionalità a un’altra, nascondendo i dettagli di implementazione interni}
}

\newglossaryentry{leadtime}{
    name={lead time},
    sort=lead time,
    description={Il termine \textit{lead time} indica l'intervallo di tempo che intercorre tra l'avvio e il completamento di un processo produttivo o logistico, utilizzato per misurare l'efficienza operativa e i tempi di risposta del sistema.}
}

\newglossaryentry{PoC}{
    name={PoC},
    text={Proof of Concept},
    sort=poc,
    description={Versione preliminare di un progetto o prototipo realizzato al fine di dimostrare la fattibilità di un’idea o di una tecnologia prima dello sviluppo completo.}
}

\newglossaryentry{KPI}{
    name={KPI},
    text={Key Performance Indicator},
    sort=kpi,
    description={Metrica quantitativa utilizzata per valutare l’efficacia di un processo o di un’attività aziendale in rapporto agli obiettivi prefissati, supportando decisioni basate su dati.}
}

\newglossaryentry{pmig}{
    name={PMI},
    text={Piccola e Media Impresa},
    sort=pmi,
    description={Impresa che, ai sensi della Raccomandazione 2003/361/CE della Commissione europea, impiega meno di 250 addetti e realizza un fatturato annuo non superiore a 50\,milioni di euro oppure un totale di bilancio annuo non superiore a 43\,milioni di euro.}
}

\makeatletter
\@ifundefined{g__tbl_row_int}{
  \expandafter\newcount\csname g__tbl_row_int\endcsname
  \csname g__tbl_row_int\endcsname=0\relax
}{}
\makeatother


% Define custom colors
\definecolor{hyperColor}{HTML}{0B3EE3}
\definecolor{tableGray}{HTML}{F5F5F7}
\definecolor{veryPeri}{HTML}{6667ab}

% Set line height
\linespread{1.5}

% Custom hyphenation rules
\hyphenation {
    data-base
    al-go-rithms
    soft-ware
}

% Images path
\graphicspath{{img/}}

% Force page color, as some editors set a grayish color as default
\pagecolor{white}

% Better spacing for footnotes
\setlength{\skip\footins}{5mm}
\setlength{\footnotesep}{5mm}

\setlength{\headheight}{14.5pt}
\addtolength{\topmargin}{-2.45pt}

% Add a subscript G to a glossary entry
\newcommand{\glox}{\textsubscript{\textbf{\textit{G}}}}

% Improvements the paragraph command
\titleformat{\paragraph}
{\normalfont\normalsize\bfseries}{\theparagraph}{1em}{}
\titlespacing*{\paragraph}
{0pt}{3.25ex plus 1ex minus .2ex}{1.5ex plus .2ex}

% Define use case environment
\newcounter{usecasecounter} % define a counter
\setcounter{usecasecounter}{0} % set the counter to some initial value
% Parameters
% #1: ID
% #2: Nome
\newenvironment{usecase}[2]{
    \renewcommand{\theusecasecounter}{\usecasename #1}  % this is where the display of the counter is overwritten/modified
    \refstepcounter{usecasecounter} % increment counter
    \vspace{2em}
    \par \noindent % start new paragraph
    {\normalsize \textbf{\usecasename #1: #2}} % display the title before the content of the environment is displayed
    \vspace{.5em}
}{
    \medskip
}
\newcommand{\usecasename}{UC}
\newcommand{\usecaseactors}[1]{\textbf{\\Attori Principali:} #1}
\newcommand{\usecasepre}[1]{\textbf{\\Precondizioni:} #1}
\newcommand{\usecasedesc}[1]{\textbf{\\Descrizione:} #1}
\newcommand{\usecasepost}[1]{\textbf{\\Postcondizioni:} #1}
\newcommand{\usecasealt}[1]{\textbf{\\Scenario Alternativo:} #1}

% Define risks environment
\newcounter{riskcounter} % define a counter
\setcounter{riskcounter}{0} % set the counter to some initial value
% Parameters
% #1: Title
\newenvironment{risk}[1]{
    \refstepcounter{riskcounter} % increment counter
    \par \noindent % start new paragraph
    \textbf{\arabic{riskcounter}. #1} % display the title before the content of the environment is displayed
    \begin{quote} % Utilizza un ambiente quote per il contenuto
}{
    \end{quote}
    \par\medskip
}
\newcommand{\riskname}{Rischio}
\newcommand{\riskdescription}[1]{\par\noindent\textbf{Descrizione:} #1.}
\newcommand{\risksolution}[1]{\par\noindent\textbf{Soluzione:} #1.}

% Apply fancy styling to pages
\pagestyle{fancy}
\fancyhf{}
\fancyhead[L]{\leftmark} % Places Chapter N. Chatper title on the top left
\fancyfoot[C]{\thepage} % Page number in the center of the footer

% Adds a blank page while increasing the page number
\newcommand\blankpage{ 
\clearpage
    \begingroup
    \null
    \thispagestyle{empty}
    \hypersetup{pageanchor=false}
    \clearpage
\endgroup
}

% Adds a blank page while increasing the page number
\newcommand\blankpagewithnumber{ 
  \clearpage
  \thispagestyle{plain} % Use plain page style to keep the page number
  \null
  \clearpage
}

% Increase page numbering
\newcommand\increasepagenumbering{
    \addtocounter{page}{+1}
}

% Make glossaries and bibliography
\makeglossaries
% Redefine the format for the glossary entries to be italic
\renewcommand*{\glstextformat}[1]{\textit{#1}\glox}
%\glsaddall

\bibliography{references/bibliography}
\defbibheading{bibliography} {
    \cleardoublepage
    \phantomsection
    \addcontentsline{toc}{chapter}{\bibname}
    \chapter*{\bibname\markboth{\bibname}{\bibname}}
}

% Code blocks handling w/ table of codes
\makeatletter
\ifdefined\NR@chapter
  \expandafter\@firstoftwo
\else
  \expandafter\@secondoftwo
\fi{\patchcmd\NR@chapter}{\patchcmd\@chapter}
  {\addtocontents{lot}{\protect\addvspace{10\p@}}}
  {\addtocontents{lot}{\protect\addvspace{10\p@}}%
   \addtocontents{lol}{\protect\addvspace{10\p@}}}
  {}{}
\makeatother

\renewcommand\listingscaption{Codice}
\renewcommand\listoflistingscaption{Elenco dei codici sorgenti}
\counterwithin*{listing}{chapter}
\renewcommand\thelisting{\thechapter.\arabic{listing}}

% Set up hyperlinks
\hypersetup{
    colorlinks=true,
    linktocpage=true,
    pdfstartpage=1,
    pdfstartview=,
    breaklinks=true,
    pdfpagemode=UseNone,
    pageanchor=true,
    pdfpagemode=UseOutlines,
    plainpages=false,
    bookmarksnumbered,
    bookmarksopen=true,
    bookmarksopenlevel=1,
    hypertexnames=true,
    pdfhighlight=/O,
    allcolors = hyperColor
}

% Set up captions
\captionsetup{
    tableposition=top,
    figureposition=bottom,
    font=small,
    format=hang,
    labelfont=bf
}

% When draft mode is on, the hyperlinks are removed. Useful when printing the document. To enable/disable, uncomment/comment the command
% \hypersetup{draft}



% Break lines in code blocks whe using inputminted
\setminted{breaklines}