% Capitolo di Introduzione — Tesi triennale di Informatica
\chapter{Introduzione}
\label{chap:introduzione}

Negli ultimi anni il settore manifatturiero è stato attraversato da una profonda trasformazione, nota con il paradigma dell’\textit{Industria~4.0}, che 
incorpora connettività diffusa, analisi dei dati in tempo reale e sistemi intelligenti di supporto alle decisioni. In questo contesto la capacità di monitorare con 
precisione le performance produttive, code ai macchinari, \gls{leadtime} e colli di bottiglia costituisce un vantaggio competitivo decisivo, soprattutto per le \gls{pmi} che puntano a ridurre gli sprechi e incrementare la flessibilità operativa.

La presente tesi nasce dalla collaborazione con l’azienda \textit{Devess~S.r.l.}, impegnata nello sviluppo di un gestionale innovativo. Durante un tirocinio di 300~ore è 
stato realizzato un \textit{proof‑of‑concept} (\gls{PoC}) di dashboard gestionale \textit{cross‑platform}, sviluppato in Flutter e sostenuto da un back‑end Python/FastAPI 
con database MongoDB, capace di integrare le \gls{API} di un \gls{LLM} per suggerire automaticamente azioni correttive o ottimizzazioni dei processi produttivi.

Per raggiungere tali traguardi è stato adottato un approccio iterativo in otto macro‑fasi settimanali, dall’onboarding tecnico al rilascio del PoC, con momenti di revisione 
periodica con il tutor aziendale. Ciascuna fase ha previsto attività di studio, sviluppo, testing e documentazione, con un focus particolare sugli algoritmi di rilevazione delle anomalie.

Oltre a costituire un contributo tangibile all’evoluzione del gestionale di Devess, il lavoro di tesi dimostra come la sinergia tra Flutter, Python e LLM abiliti soluzioni 
agili e scalabili per la \textit{smart manufacturing}, favorendo decisioni proattive e riducendo il \textit{time‑to‑action} dei responsabili di produzione.

\section{L'azienda}
Fondata nel 2020, \textit{Devess~S.r.l.} ha sede in Via Per Curnasco 58, Bergamo, e si occupa di consulenza e realizzazione di software gestionali per il settore manifatturiero. 
La piattaforma proprietaria \textit{Operativo.io} consente alle aziende di monitorare in tempo reale il flusso produttivo, analizzare li \textit{Key Performance Indicators} 
(\gls{KPI}) e intervenire prontamente in caso di inefficienze; proprio in questo software verrà implementata la nuova dashboard.

La mission di Devess è migliorare la gestione degli ordini nelle \gls{pmi}, coniugando l’esperienza dei processi produttivi tradizionali con le potenzialità offerte 
dal digitale e l'industria 4.0. L’orientamento alla \textit{data‑driven decision making} sta rendendo 
l’azienda una valida soluzione per le \gls{pmi} che desiderano intraprendere un percorso di trasformazione digitale, contenendo tempi e costi di adozione.

Durante il tirocinio il tutor aziendale, Nicola Romano, ha coordinato incontri settimanali — sia in presenza sia in modalità telematica — per monitorare l’avanzamento, fornire 
feedback puntuali e garantire l’allineamento agli obiettivi di progetto O01–O05 definiti nel piano di lavoro.

\section{L'idea}
Il progetto di stage si proponeva di sviluppare un sistema dimostrativo che integrasse analisi dati, visualizzazione interattiva e suggerimenti generati da un modello linguistico 
di grandi dimensioni. L’idea era duplice:
\begin{enumerate}
\item \textbf{Visualizzazione delle criticità}. Realizzare una dashboard che mettesse in evidenza code alle macchine, \gls{leadtime} e colli di bottiglia trovati tramite l'analisi dei dati presenti nel database MongoDB.
\item \textbf{Supporto decisionale intelligente}. Sfruttare un \gls{LLM} per suggerire, alla luce dei dati analizzati, azioni correttive concrete con potenziale estensione a un chatbot.
\end{enumerate}

La combinazione di queste componenti permette al management di individuare una criticità in pochi clic e di ricevere, contestualmente, soluzioni suggerite dall’intelligenza artificiale. Tale approccio, oltre a ridurre i tempi di analisi, favorisce una cultura operativa basata su \textit{insight} immediati e misurabili.



\section{Organizzazione del testo}
\begin{description}
    \item[{\hyperref[chap:descrizione-stage]{Il secondo capitolo}}] documenta l’impostazione operativa dello stage, le modalità di interazione con l’azienda e con il tutor, l’approccio metodologico adottato e l’analisi dei principali rischi.

    \item[{\hyperref[chap:analisi-requisiti]{Il terzo capitolo}}] presenta un’esame approfondito dei requisiti, suddivisi in obbligatori, desiderabili e opzionali, corredato dalla mappatura di attori, \textit{stakeholder}, \textit{use case} e funzionalità previste.

    \item[{\hyperref[chap:introduzione-teorica]{Il quarto capitolo}}] espone i fondamenti teorici e tecnologici alla base del progetto, confronta le soluzioni esistenti e argomenta la scelta degli strumenti impiegati.

    \item[{\hyperref[chap:lavoro-svolto]{Il quinto (ed eventuali sesto e settimo) capitolo}}] descrive in dettaglio le attività eseguite, le criticità incontrate e le soluzioni adottate, includendo schemi, tabelle e immagini ove opportuno.

    \item[{\hyperref[chap:conclusioni]{Il capitolo conclusivo}}] raccoglie le riflessioni finali: valutazione del raggiungimento degli obiettivi, competenze acquisite, possibili sviluppi futuri e considerazioni personali.
\end{description}




Riguardo alla stesura del testo, relativamente al documento sono state adottate le seguenti convenzioni tipografiche:
\begin{itemize}
  \item gli acronimi, le abbreviazioni e i termini ambigui o di uso non comune menzionati vengono definiti nel glossario, situato alla fine del presente documento;
  \item per la prima occorrenza dei termini riportati nel glossario viene utilizzata la seguente nomenclatura: \gls{apig};
  \item i termini in lingua straniera o facenti parti del gergo tecnico sono evidenziati con il carattere \textit{corsivo}.
\end{itemize}

\newpage
