% Capitolo 2 — Descrizione dello stage
\chapter{Descrizione dello stage}
\label{chap:descrizione-stage}

\section{Introduzione al progetto}

Negli ultimi anni il settore manifatturiero ha intrapreso un percorso di digitalizzazione incentrato sul paradigma dell’\textit{Industria~4.0}. In tale contesto il presente 
tirocinio, svolto presso l’azienda \textit{Devess~S.r.l.}, si propone di realizzare un \textit{proof‑of‑concept} (\gls{PoC}) da implementare in un software gestionale 
\textit{cross‑platform} sviluppato con Flutter, supportato da un back‑end \textit{Python}/FastAPI con persistenza in MongoDB e integrazione di un modello linguistico di grandi 
dimensioni (\gls{LLM}). L’obiettivo primario è consentire al management di individuare tempestivamente criticità operative, code ai macchinari, \gls{leadtime}, anomalie, colli di 
bottiglia e di ricevere suggerimenti automatici di ottimizzazione basati sui dati.

Il progetto si articola in otto macro‑fasi (cfr. sezione~\ref{subsec:pianificazione}), per un totale di 300~ore, e combina attività di analisi dati, \textit{prompt engineering}, 
sviluppo \textit{mobile} e validazione sperimentale. Il tutor aziendale, Nicola Romano, ha seguito l’avanzamento mediante allineamenti settimanali sui risultati ottenuti, 
facilitando l’accesso alle risorse necessarie.

\section{Analisi preventiva dei rischi}

\begin{risk}{Qualità dei dati in MongoDB}
\riskdescription{I dati storici potrebbero presentare valori mancanti o incoerenti, compromettendo l’accuratezza delle analisi e dei suggerimenti generati dall’intelligenza artificiale}
\risksolution{Definire procedure di \textit{data cleaning} automatizzate e introdurre controlli di validazione a campione}
\end{risk}

\begin{risk}{Costi e limiti di utilizzo delle \gls{API} LLM}
\riskdescription{L’uso intensivo delle \gls{API} di ChatGPT potrebbe superare il budget assegnato o incorrere in \textit{rate‑limit}, rendendo la funzionalità economicamente insostenibile}
\risksolution{Svolgere un’accurata analisi dei costi e monitorare i consumi del prototipo tramite \textit{dashboard} dedicata, al fine di stimare il costo medio d’utilizzo e adeguare di conseguenza il pricing del software}
\end{risk}

\begin{risk}{non-determinismo delle risposte da parte del LLM}
\riskdescription{La natura probabilistica del \gls{LLM} può produrre risposte diverse a parità di input, introducendo possibili incoerenze o inesattezze nei suggerimenti operativi}
\risksolution{Utilizzare istruzioni precise tramite \textit{prompt engineering}, e prevedere un workflow di approvazione manuale per le decisioni critiche}
\end{risk}

\begin{risk}{Scalabilità del back‑end}
\riskdescription{L’aumento del volume di dati o di richieste simultanee potrebbe ridurre le prestazioni degli endpoint FastAPI}
\risksolution{Progettare l’architettura con attenzione alla scalabilità, selezionare il deployment più adatto a un utilizzo su larga scala, adottare containerizzazione e bilanciamento del carico, prevedendo la replica del database e test di \textit{stress}}
\end{risk}

\begin{risk}{Usabilità dell’interfaccia Flutter}
\riskdescription{Un’interfaccia poco intuitiva potrebbe vanificare i benefici dell’automazione, rallentando l’adozione da parte del management}
\risksolution{Applicare le linee guida Material~3, condurre \textit{usability test} con utenti interni e iterare sulle funzionalità maggiormente critiche}
\end{risk}

\newpage

\section{Requisiti e obiettivi}

\begin{center}
    \rowcolors{1}{}{tableGray}
    \begin{tabular}{|p{2.25cm}|p{12.00cm}|}
    \hline
    \rowcolor{tableGray}\textbf{Codice} & \textbf{Descrizione} \\
    \hline
    O01 & Dashboard Flutter che visualizzi code dei macchinari, \gls{leadtime} e colli di bottiglia in tempo reale, integrata nel gestionale mobile di Devess.\\
    \hline
    O02 & Integrazione di GPT come \gls{LLM} per restituire risposte e suggerimenti in linguaggio naturale basati sui dati.\\
    \hline
    O03 & Implementazione di funzioni analitiche in grado di identificare anomalie nei processi produttivi.\\
    \hline
    O04 & Protezione dei dati sensibili con meccanismi di cifratura e controllo degli accessi basato su ruoli.\\
    \hline
    O05 & Interfaccia utente studiata in modo accurato, fruibile anche da operatori non specializzati.\\
    \hline
    D01 & Realizzazione di una versione \textit{web} dell’applicazione, integrata nel gestionale Devess.\\
    \hline
    D02 & Possibilità di interagire con un chatbot basato su \gls{LLM} per richiedere chiarimenti o soluzioni mirate.\\
    \hline
    D03 & Notifiche push o e‑mail proattive al manifestarsi di problemi o rallentamenti nella produzione.\\
    \hline
    F01 & Sincronizzazione offline con riallineamento automatico al ripristino della connettività.\\
    \hline
    F02 & Animazioni fluide e tempi di caricamento percepiti veloci.\\
    \hline
    \end{tabular}
    \captionof{table}{Requisiti e obiettivi definiti per il tirocinio.}
    \label{tab:requisiti_obbiettivi}
\end{center}

\section{Pianificazione}
\label{subsec:pianificazione}

Il lavoro è suddiviso in otto settimane; ciascuna prevede attività specifiche, \textit{milestone} e momenti di verifica con il tutor aziendale.

\begin{enumerate}
    \item \textbf{Settimana~1 — Onboarding e configurazione ambienti (40~h)}
        \begin{itemize}
            \item Configurazione degli ambienti \textit{Python}/FastAPI, Flutter e MongoDB.
            \item Studio delle \gls{API} di ChatGPT e definizione degli \textit{use case} AI.
        \end{itemize}
    \item \textbf{Settimana~2 — Prompt engineering e integrazione base LLM (40~h)}
        \begin{itemize}
            \item Applicazione di tecniche di \textit{prompt engineering}.
            \item Prima integrazione delle \gls{API} LLM con tracciamento di costi e latenza.
        \end{itemize}
    \item \textbf{Settimana~3 — Modellazione dati e MongoDB (40~h)}
        \begin{itemize}
            \item Definizione dello schema NoSQL per macchinari e produzione.
            \item Importazione e indicizzazione dei dati aziendali, con politiche di sicurezza.
        \end{itemize}
    \item \textbf{Settimana~4 — Back‑end Python e algoritmi analitici (40~h)}
        \begin{itemize}
            \item Realizzazione degli endpoint CRUD.
            \item Sviluppo degli algoritmi per code, \gls{leadtime} e colli di bottiglia.
        \end{itemize}
    \item \textbf{Settimana~5 — Validazione algoritmi e suggerimenti AI (40~h)}
        \begin{itemize}
            \item Valutazione del corretto funzionamento degli algoritmi di analisi dati.
            \item Integrazione con ChatGPT per suggerimenti contestuali.
        \end{itemize}
    \item \textbf{Settimana~6 — Automazione e performance (40~h)}
        \begin{itemize}
            \item Scheduler per analisi periodiche.
            \item Ottimizzazione delle query e monitoraggio dei costi operativi.
        \end{itemize}
    \item \textbf{Settimana~7 — UI e dashboard Flutter (40~h)}
        \begin{itemize}
            \item Progettazione del \textit{design system} secondo Material~3.
            \item Implementazione degli schermi di monitoraggio e test multi‑device.
        \end{itemize}
    \item \textbf{Settimana~8 — QA, MLOps e rilascio (20~h)}
        \begin{itemize}
            \item Test end‑to‑end e preparazione del pacchetto di rilascio.
            \item Stesura della documentazione tecnica conclusiva.
        \end{itemize}
\end{enumerate}

\subsection{Modalità di interazione con il tutor aziendale}

Sono previsti incontri di avanzamento con il tutor ogni settimana, in presenza o da remoto, per discutere le criticità emerse, verificare il rispetto del piano e concordare eventuali adeguamenti.

\subsection{Deliverable previsti}

\begin{itemize}
    \item \textbf{PoC} della dashboard gestionale completa di analisi e suggerimenti AI.
    \item Report tecnico finale strutturato nei capitoli di introduzione, architettura, metodologie, risultati e conclusioni.
    \item Codice sorgente versionato su repository Git proprietaria.
\end{itemize}

\newpage
